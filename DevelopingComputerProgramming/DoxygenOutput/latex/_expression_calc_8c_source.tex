\hypertarget{_expression_calc_8c_source}{\section{Expression\+Calc.\+c}
\label{_expression_calc_8c_source}\index{Developing\+Programming/\+Developing\+Computer\+Programming/\+Expression\+Calc.\+c@{Developing\+Programming/\+Developing\+Computer\+Programming/\+Expression\+Calc.\+c}}
}

\begin{DoxyCode}
00001 \textcolor{preprocessor}{#include <stdio.h>}
00002 \textcolor{preprocessor}{#include <stdlib.h>}
00003 
00004 \textcolor{comment}{//Function prototype to calculate Fibonacci series}
00005 \textcolor{keywordtype}{void} \hyperlink{_expression_calc_8c_a79cd2ba7236d8017474e3340e169f888}{Extra}();
00006 \textcolor{keywordtype}{void} \hyperlink{_expression_calc_8c_a1f044085ede204d3d2ba2e4db44e7e5d}{CalcFib}(\textcolor{keywordtype}{int} r);                                             
00007 
\hypertarget{_expression_calc_8c_source_l00008}{}\hyperlink{_expression_calc_8c_ae66f6b31b5ad750f1fe042a706a4e3d4}{00008} \textcolor{keywordtype}{int} \hyperlink{_expression_calc_8c_ae66f6b31b5ad750f1fe042a706a4e3d4}{main}()
00009 \{
00010   \textcolor{comment}{//Welcome Screen}
00011   printf(\textcolor{stringliteral}{"Welcome to Calculator. \(\backslash\)n"});                          
00012   \textcolor{keywordtype}{char} check;
00013   \textcolor{keywordtype}{int} r;
00014   \textcolor{keywordtype}{int} ch;
00015   \textcolor{keywordtype}{char} opt;
00016   \textcolor{keywordtype}{float} var1, var2;
00017   \textcolor{keywordtype}{float} res;
00018   FILE * Log;
00019   
00020   \textcolor{comment}{//Opening a new file for Logging info. Opening in reading and writing mode}
00021   Log = fopen (\textcolor{stringliteral}{"Log.txt"}, \textcolor{stringliteral}{"w+"});                            
00022       
00023     \textcolor{keywordflow}{do}
00024     \{                                                           
00025       printf(\textcolor{stringliteral}{"Please select an Arithmetic operation from the following: \(\backslash\)n"});           
00026       printf(\textcolor{stringliteral}{"1)Addition \(\backslash\)t 2)Subtraction \(\backslash\)t 3)Multiplication \(\backslash\)t 4)Division \(\backslash\)t 5)Extra Features \(\backslash\)n "});
00027       
00028       \textcolor{comment}{//Input of operation, A SPACE before %d makes scanf ignore whitespace}
00029       scanf(\textcolor{stringliteral}{" %d"}, &ch);                                            
00030       
00031       \textcolor{keywordflow}{if}(ch!=5)
00032       \{
00033         \textcolor{comment}{//Asking user if he wants to use saved previous result}
00034         printf(\textcolor{stringliteral}{"Do you want to use the previous result as one of the operands (Y/N)? \(\backslash\)n"});
00035         scanf(\textcolor{stringliteral}{" %c"}, &opt);                                                 
00036       
00037         printf(\textcolor{stringliteral}{"Enter any values for operand(s): \(\backslash\)n"});
00038         \textcolor{keywordflow}{if}(opt==\textcolor{charliteral}{'Y'} || opt==\textcolor{charliteral}{'y'})
00039           \{
00040             printf(\textcolor{stringliteral}{"Operand 1 is the previous result: "});
00041             rewind(Log);
00042             \textcolor{comment}{//Taking value from the LOG text file for variable 1}
00043             fscanf(Log, \textcolor{stringliteral}{" %f"}, &var1);                                  
00044             printf(\textcolor{stringliteral}{"%f \(\backslash\)n"}, var1);
00045           \}  
00046       
00047         \textcolor{keywordflow}{else} 
00048         \{
00049           printf(\textcolor{stringliteral}{"Operand 1: "}); 
00050           scanf(\textcolor{stringliteral}{"%f"}, &var1);                                       
00051         \}
00052         
00053         printf(\textcolor{stringliteral}{"Operand 2: "}); 
00054         \textcolor{comment}{/* An optional starting asterisk or space indicates }
00055 \textcolor{comment}{        *that the data is to be read from the stream but }
00056 \textcolor{comment}{        *ignored, i.e. it is not stored in the corresponding argument.  }
00057 \textcolor{comment}{        *Input of variable2 */}
00058         
00059         scanf(\textcolor{stringliteral}{" %f"}, &var2);
00060         
00061         \textcolor{comment}{//Switch case construct: Statements for different conditions }
00062         \textcolor{keywordflow}{switch}(ch)                                              
00063         \{
00064           \textcolor{keywordflow}{case} 1: res=var1+var2; \textcolor{keywordflow}{break};
00065           \textcolor{keywordflow}{case} 2: res=var1-var2; \textcolor{keywordflow}{break};
00066           \textcolor{keywordflow}{case} 3: res=var1*var2; \textcolor{keywordflow}{break};
00067           \textcolor{keywordflow}{case} 4: res=var1/var2; \textcolor{keywordflow}{break};
00068         \}
00069          
00070         \textcolor{comment}{//Displaying result from file}
00071         printf(\textcolor{stringliteral}{"\(\backslash\)nThe Result= %f \(\backslash\)n"}, res);                         
00072       
00073         rewind(Log);
00074         \textcolor{comment}{//Logging the result}
00075         fprintf(Log, \textcolor{stringliteral}{" %f"}, res);                               
00076       
00077       \}
00078       
00079       \textcolor{comment}{//Calling the CalcFib() function and passing range as parameter.}
00080       \textcolor{keywordflow}{else} \hyperlink{_expression_calc_8c_a79cd2ba7236d8017474e3340e169f888}{Extra}();                                     
00081       
00082       \textcolor{comment}{//Asking user to continue or exit}
00083       printf(\textcolor{stringliteral}{"Continue y/n?  "});                                
00084       scanf(\textcolor{stringliteral}{" %c"},&check);                                      
00085       
00086     \}\textcolor{keywordflow}{while}(check!=\textcolor{charliteral}{'n'});                 
00087    
00088     \textcolor{comment}{// Closing opened text file}
00089     fclose(Log);                                                    
00090    
00091 \textcolor{keywordflow}{return}(0);
00092 \}
00093 
00094 
\hypertarget{_expression_calc_8c_source_l00096}{}\hyperlink{_expression_calc_8c_a79cd2ba7236d8017474e3340e169f888}{00096} \textcolor{keywordtype}{void} \hyperlink{_expression_calc_8c_a79cd2ba7236d8017474e3340e169f888}{Extra}()                                               
00097 \{ 
00098   \textcolor{keywordtype}{int} ch,r;
00099   printf(\textcolor{stringliteral}{"\(\backslash\)nSelect from the following options: \(\backslash\)n1)Fibonacci Series \(\backslash\)n"});
00100   scanf(\textcolor{stringliteral}{" %d"},&ch);
00101   \textcolor{keywordflow}{switch}(ch)
00102   \{
00103     \textcolor{keywordflow}{case} 1: printf(\textcolor{stringliteral}{"Enter the range of Fibonacci series: \(\backslash\)n "}); scanf(\textcolor{stringliteral}{" %d"},&r); 
      \hyperlink{_expression_calc_8c_a1f044085ede204d3d2ba2e4db44e7e5d}{CalcFib}(r); \textcolor{keywordflow}{break};
00104   \}
00105 \}
00106 
00107 
\hypertarget{_expression_calc_8c_source_l00109}{}\hyperlink{_expression_calc_8c_a1f044085ede204d3d2ba2e4db44e7e5d}{00109} \textcolor{keywordtype}{void} \hyperlink{_expression_calc_8c_a1f044085ede204d3d2ba2e4db44e7e5d}{CalcFib}(\textcolor{keywordtype}{int} r)                                          
00110 \{
00111   \textcolor{keywordtype}{int} j=0, k=1, res;
00112   printf(\textcolor{stringliteral}{"FIBONACCI SERIES: "});
00113   
00114   \textcolor{comment}{//printing first two values.}
00115   printf(\textcolor{stringliteral}{"%d %d"},j,k); 
00116 
00117     \textcolor{keywordflow}{for}(\textcolor{keywordtype}{int} i=2;i<r;i++)
00118     \{
00119       res=j+k;
00120       j=k;
00121       k=res; 
00122       printf(\textcolor{stringliteral}{" %d"},k);
00123     \}  
00124   printf(\textcolor{stringliteral}{"\(\backslash\)n"}); 
00125 \}
00126 
\end{DoxyCode}
